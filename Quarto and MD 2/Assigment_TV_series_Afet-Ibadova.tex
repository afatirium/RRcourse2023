% Options for packages loaded elsewhere
\PassOptionsToPackage{unicode}{hyperref}
\PassOptionsToPackage{hyphens}{url}
\PassOptionsToPackage{dvipsnames,svgnames,x11names}{xcolor}
%
\documentclass[
  letterpaper,
  DIV=11,
  numbers=noendperiod]{scrartcl}

\usepackage{amsmath,amssymb}
\usepackage{lmodern}
\usepackage{iftex}
\ifPDFTeX
  \usepackage[T1]{fontenc}
  \usepackage[utf8]{inputenc}
  \usepackage{textcomp} % provide euro and other symbols
\else % if luatex or xetex
  \usepackage{unicode-math}
  \defaultfontfeatures{Scale=MatchLowercase}
  \defaultfontfeatures[\rmfamily]{Ligatures=TeX,Scale=1}
\fi
% Use upquote if available, for straight quotes in verbatim environments
\IfFileExists{upquote.sty}{\usepackage{upquote}}{}
\IfFileExists{microtype.sty}{% use microtype if available
  \usepackage[]{microtype}
  \UseMicrotypeSet[protrusion]{basicmath} % disable protrusion for tt fonts
}{}
\makeatletter
\@ifundefined{KOMAClassName}{% if non-KOMA class
  \IfFileExists{parskip.sty}{%
    \usepackage{parskip}
  }{% else
    \setlength{\parindent}{0pt}
    \setlength{\parskip}{6pt plus 2pt minus 1pt}}
}{% if KOMA class
  \KOMAoptions{parskip=half}}
\makeatother
\usepackage{xcolor}
\setlength{\emergencystretch}{3em} % prevent overfull lines
\setcounter{secnumdepth}{2}
% Make \paragraph and \subparagraph free-standing
\ifx\paragraph\undefined\else
  \let\oldparagraph\paragraph
  \renewcommand{\paragraph}[1]{\oldparagraph{#1}\mbox{}}
\fi
\ifx\subparagraph\undefined\else
  \let\oldsubparagraph\subparagraph
  \renewcommand{\subparagraph}[1]{\oldsubparagraph{#1}\mbox{}}
\fi


\providecommand{\tightlist}{%
  \setlength{\itemsep}{0pt}\setlength{\parskip}{0pt}}\usepackage{longtable,booktabs,array}
\usepackage{calc} % for calculating minipage widths
% Correct order of tables after \paragraph or \subparagraph
\usepackage{etoolbox}
\makeatletter
\patchcmd\longtable{\par}{\if@noskipsec\mbox{}\fi\par}{}{}
\makeatother
% Allow footnotes in longtable head/foot
\IfFileExists{footnotehyper.sty}{\usepackage{footnotehyper}}{\usepackage{footnote}}
\makesavenoteenv{longtable}
\usepackage{graphicx}
\makeatletter
\def\maxwidth{\ifdim\Gin@nat@width>\linewidth\linewidth\else\Gin@nat@width\fi}
\def\maxheight{\ifdim\Gin@nat@height>\textheight\textheight\else\Gin@nat@height\fi}
\makeatother
% Scale images if necessary, so that they will not overflow the page
% margins by default, and it is still possible to overwrite the defaults
% using explicit options in \includegraphics[width, height, ...]{}
\setkeys{Gin}{width=\maxwidth,height=\maxheight,keepaspectratio}
% Set default figure placement to htbp
\makeatletter
\def\fps@figure{htbp}
\makeatother

\KOMAoption{captions}{tableheading}
\makeatletter
\makeatother
\makeatletter
\makeatother
\makeatletter
\@ifpackageloaded{caption}{}{\usepackage{caption}}
\AtBeginDocument{%
\ifdefined\contentsname
  \renewcommand*\contentsname{Table of contents}
\else
  \newcommand\contentsname{Table of contents}
\fi
\ifdefined\listfigurename
  \renewcommand*\listfigurename{List of Figures}
\else
  \newcommand\listfigurename{List of Figures}
\fi
\ifdefined\listtablename
  \renewcommand*\listtablename{List of Tables}
\else
  \newcommand\listtablename{List of Tables}
\fi
\ifdefined\figurename
  \renewcommand*\figurename{Figure}
\else
  \newcommand\figurename{Figure}
\fi
\ifdefined\tablename
  \renewcommand*\tablename{Table}
\else
  \newcommand\tablename{Table}
\fi
}
\@ifpackageloaded{float}{}{\usepackage{float}}
\floatstyle{ruled}
\@ifundefined{c@chapter}{\newfloat{codelisting}{h}{lop}}{\newfloat{codelisting}{h}{lop}[chapter]}
\floatname{codelisting}{Listing}
\newcommand*\listoflistings{\listof{codelisting}{List of Listings}}
\makeatother
\makeatletter
\@ifpackageloaded{caption}{}{\usepackage{caption}}
\@ifpackageloaded{subcaption}{}{\usepackage{subcaption}}
\makeatother
\makeatletter
\@ifpackageloaded{tcolorbox}{}{\usepackage[many]{tcolorbox}}
\makeatother
\makeatletter
\@ifundefined{shadecolor}{\definecolor{shadecolor}{rgb}{.97, .97, .97}}
\makeatother
\makeatletter
\makeatother
\ifLuaTeX
  \usepackage{selnolig}  % disable illegal ligatures
\fi
\IfFileExists{bookmark.sty}{\usepackage{bookmark}}{\usepackage{hyperref}}
\IfFileExists{xurl.sty}{\usepackage{xurl}}{} % add URL line breaks if available
\urlstyle{same} % disable monospaced font for URLs
\hypersetup{
  pdftitle={Seinfeld},
  pdfauthor={Afet Ibadova},
  colorlinks=true,
  linkcolor={blue},
  filecolor={Maroon},
  citecolor={Blue},
  urlcolor={Blue},
  pdfcreator={LaTeX via pandoc}}

\title{Seinfeld}
\author{Afet Ibadova}
\date{4/20/23}

\begin{document}
\maketitle
\ifdefined\Shaded\renewenvironment{Shaded}{\begin{tcolorbox}[sharp corners, interior hidden, breakable, enhanced, borderline west={3pt}{0pt}{shadecolor}, boxrule=0pt, frame hidden]}{\end{tcolorbox}}\fi

\listoffigures
\listoftables
\begin{center}\rule{0.5\linewidth}{0.5pt}\end{center}

\hypertarget{description-about-seinfeld}{%
\section{\texorpdfstring{Description about
\emph{Seinfeld}}{Description about Seinfeld}}\label{description-about-seinfeld}}

\hypertarget{details-about-the-show}{%
\subsection{Details about the show}\label{details-about-the-show}}

\begin{quote}
\textbf{Seinfeld} is an American television sitcom created by
\textbf{Jerry Seinfeld} and \textbf{Larry David}. Seinfeld has been
described by some as a ``show about nothing'', similar to the
self-parodying ``show within a show'' of fourth-season episode ``The
Pilot''. Jerry Seinfeld is the lead character and played as a
fictionalized version of himself. Set predominantly in an apartment
block on New York City's Upper West Side, the show features a host of
Jerry's friends and acquaintances, which include \emph{George Costanza},
\emph{Elaine Benes}, and \emph{Cosmo Kramer}, who are portrayed by
\emph{Jason Alexander}, \emph{Julia Louis-Dreyfus}, and \emph{Michael
Richards}, respectively.
\end{quote}

\hypertarget{the-official-logo-of-the-tv-series}{%
\subsection{The official logo of the TV
Series}\label{the-official-logo-of-the-tv-series}}

\includegraphics{https://upload.wikimedia.org/wikipedia/commons/thumb/7/78/Seinfeld_logo.svg/345px-Seinfeld_logo.svg.png}\\

\hypertarget{basic-statistics}{%
\subsection{Basic Statistics}\label{basic-statistics}}

\begin{longtable}[]{@{}
  >{\centering\arraybackslash}p{(\columnwidth - 12\tabcolsep) * \real{0.0933}}
  >{\centering\arraybackslash}p{(\columnwidth - 12\tabcolsep) * \real{0.1067}}
  >{\centering\arraybackslash}p{(\columnwidth - 12\tabcolsep) * \real{0.2000}}
  >{\centering\arraybackslash}p{(\columnwidth - 12\tabcolsep) * \real{0.2133}}
  >{\centering\arraybackslash}p{(\columnwidth - 12\tabcolsep) * \real{0.0800}}
  >{\centering\arraybackslash}p{(\columnwidth - 12\tabcolsep) * \real{0.1067}}
  >{\centering\arraybackslash}p{(\columnwidth - 12\tabcolsep) * \real{0.2000}}@{}}
\caption{Season overview}\tabularnewline
\toprule()
\begin{minipage}[b]{\linewidth}\centering
\textbf{Season}
\end{minipage} & \begin{minipage}[b]{\linewidth}\centering
\textbf{Episodes}
\end{minipage} & \begin{minipage}[b]{\linewidth}\centering
\textbf{First aired}
\end{minipage} & \begin{minipage}[b]{\linewidth}\centering
\textbf{Last aired}
\end{minipage} & \begin{minipage}[b]{\linewidth}\centering
Rank
\end{minipage} & \begin{minipage}[b]{\linewidth}\centering
Rating
\end{minipage} & \begin{minipage}[b]{\linewidth}\centering
\textbf{Viewers(millions)}
\end{minipage} \\
\midrule()
\endfirsthead
\toprule()
\begin{minipage}[b]{\linewidth}\centering
\textbf{Season}
\end{minipage} & \begin{minipage}[b]{\linewidth}\centering
\textbf{Episodes}
\end{minipage} & \begin{minipage}[b]{\linewidth}\centering
\textbf{First aired}
\end{minipage} & \begin{minipage}[b]{\linewidth}\centering
\textbf{Last aired}
\end{minipage} & \begin{minipage}[b]{\linewidth}\centering
Rank
\end{minipage} & \begin{minipage}[b]{\linewidth}\centering
Rating
\end{minipage} & \begin{minipage}[b]{\linewidth}\centering
\textbf{Viewers(millions)}
\end{minipage} \\
\midrule()
\endhead
1 & 5 & July 5, 1989 & June 21, 1990 & - & - & 19.2 \\
2 & 12 & January 23, 1991 & June 26, 1991 & 46 & 12.5 & 18.1 \\
3 & 23 & September 18, 1991 & May 6, 1992 & 43 & 12.5 & 17.7 \\
4 & 24 & August 12,1992 & May 20, 1993 & 25 & 13.7 & 20.0 \\
5 & 22 & September 16, 1993 & May 19, 1994 & 3 & 19.4 & 29.6 \\
6 & 24 & September 22, 1994 & May 11, 1995 & 1 & 20.6 & 31.1 \\
7 & 24 & September 21, 1995 & May 16, 1996 & 2 & 21.2 & 33.1 \\
8 & 22 & September 19, 1996 & May 15, 1997 & 2 & 20.5 & 32.3 \\
9 & 24 & September 25, 1997 & May 14, 1998 & 1 & 22.0 & 35.5 \\
\bottomrule()
\end{longtable}



\end{document}
